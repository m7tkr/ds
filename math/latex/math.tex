\documentclass[a4paper,12pt]{article}
\author{m7tkr}
\usepackage[english]{babel}
\usepackage{array}
\renewcommand{\arraystretch}{1.5}
\usepackage{amsmath}
\usepackage{amssymb}

\begin{document}

\tableofcontents

\section{algebra}

\subsection{wallace}

\subsubsection{distributive property}

\[ a(b+c)=ac+bc \]

\subsubsection{slope}

\[ \mathit{m}=\frac{\textbf{rise}}{\textbf{run}}=\frac{y_2-y_1}{x_2-x_1} \]

\subsubsection{properties of exponents}
\begin{center}

        \begin{tabular}{*{3}c}
            
                \( a^{m}a^n=a^{m+n} \)                           & \(
                (ab)^m=a^{m}b^{m} \)    & \( \frac{a^m}{a^n}=a^{m-n} \) \\[10pt]
                \( \left(\frac{a}{b}\right)^m=\frac{a^m}{b^m} \) & \( a^{-m}=\frac{1}{a^m} \) & \( \frac{1}{a^{-m}}=a^m \)    \\[10pt]
                \( (a^m)^n=a^{mn} \)                             & \( a^0=1 \)                & \(
                \left(\frac{a}{b}\right)^{-m}=\frac{b^m}{a^m} \)                                                              \\
        \end{tabular}
\end{center}

\subsubsection{scientific notation}

\[ a \times 10^b \text{ where } 1 \leqslant a < 10 \]

\subsubsection{ways to factor}

\begin{itemize}
        \item GCF
        \item Grouping
        \item Trinomials where \( a = 1 \)
              \begin{itemize}
                      \item multiply to \( a \times c \)
                      \item add to \( b \)
              \end{itemize}

        \item Trinomials where \( a \neq 1 \)
        \item Factoring Special Products

\end{itemize}

\subsubsection{factoring special products}

\begin{center}

\begin{tabular}{r l}
    difference of square & \( a^{2}-b^{2}=(a-b)(a+b) \) \\
    sum of squares & \( a^{2} + b^{2} = \text{Prime} \) \\
    perfect square & \(  a^{2} + 2ab + b^{2} =  (a+b)^{2} \) \\
    sum of cubes & \( a^{3} + b^{3} = (a+b)( a^{2} -ab + b^{2} ) \) \\
    difference of cubes & \( a^{3} - b^{3} = (a-b)( a^{2} +ab + b^{2} ) \)
\end{tabular}
\end{center}

\subsubsection{factoring strategy}

\begin{itemize}
    \item \textbf{GCF FIRST}
    \item \textbf{2 terms}: sum of diffs of squares or cubes
    \item \textbf{3 terms}: ac method, watch for perfect squares
    \item \textbf{4 terms}: grouping
\end{itemize}

\subsubsection{cross product}

\[ \text{if } \frac{a}{b} = \frac{c}{d}, \text{ then } ad = bc \]

\subsubsection{definition of radicals}

\[ \sqrt[m]{a} = b, \text{ if } b^{m} = a \]

\subsubsection{properties of radicals}

\begin{center}
    \begin{tabular}{*{3}c}
        \( a^{m} a^{n} = a^{m+n} \) & \(  (ab)^{m} = a^{m} b^{m} \) & \( a^{-m}
        = \frac{1}{a^{m}} \) \\[10pt]
        \( \frac{a^{m}}{a^{n}} = a^{m-n} \) & \( \displaystyle{
            \left(\frac{a}{b}\right)^{m} } =
        \frac{a^{m}}{b^{m}} \) & \( \frac{1}{a^{-m}} = a^{m} \) \\[10pt]
        \( \left(a^{m}\right)^{n} = a^{mn} \) & \( a^{0} = 1 \) & \(  
        \left(\frac{a}{b}\right)^{-m} = \frac{b^{m}}{a^{m}} \)
    \end{tabular}\\[1cm]
\end{center}
\begin{center}
\fbox{ \textbf{ always rationalize denominator } }
\end{center}

\subsubsection{radicals of mixed index}

\subsubsection{definition of rational exponents}

\[ a^{\frac{n}{m}}=\left(\sqrt[m]{a}\right)^{n} \]

\subsubsection{definition of imaginary numbers}

\[ i^{2}=-1 \left( \text{thus } i=\sqrt{-1} \right) \]

\subsubsection{cyclic property of powers of \( i \) }

%%% math-vimtex-template.tex
\begin{alignat*}{4}
&i^{0} &&= 1 \\
&i  &&= i \\
&i^{2} &&= -1 \\
&i^{3} &&= -i
\end{alignat*}

\paragraph{Example}

\begin{alignat*}{2}
    i^{35} \quad & \text{\raggedleft divide exponent by 4} \\
    8R3 \quad & \text{use remainder as exp of }i  \\
    i^{3} \quad & \text{Simplify} \\
    -i \quad & \text{Solution}
\end{alignat*}

\begin{center}
\fbox{\parbox{9cm}{ \textbf{when solving a radical problem with an even
index: check answers} } }
\end{center}

\subsubsection{odd root property}

\[ \text{if } a^{n} = b \text{, then } a = \sqrt[n]{b} \text{ when } n \text{ is odd} \]

\subsubsection{even root property}

\[ \text{if } a^{n} = b \text{, then } a = \pm\sqrt[n]{b} \text{ when } n \text{ is even} \]
\section{geometry}
\end{document}

